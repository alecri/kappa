% ---------------------------------------------------------
% Project: PhD KAPPA
% File: sammanfattning.tex
% Author: Alessio Crippa
%
% Purpose: Sammanfattning frontmatter
% ---------------------------------------------------------


\noindent {\Large \textsf{\textbf{{Sammanfattning}}}
\bigskip
\small
\par Dos-respons metaanalys är ett statistiskt förfarande för att kombinera och jämföra resultat från epidemiologiska studier där sambandet mellan en kontinuerlig variabel och en hälsorisk har undersökts. Tidigare studier har förfinat delar av metoden, till exempel genom införande av flexibla strategier och utökning till multivariata modeller; men trots detta kvarstår flera relevanta metodologiska frågor. Denna avhandling syftar till att besvara ett antal av dessa frågor genom att utveckla och presentera: nya strategier och ad hoc-modeller (Artikel 1); nya metoder för att bedöma goodness-of-fit (Artikel 2); ett nytt mått för att kvantifiera påverkan av heterogenitet (Artikel 3); en ny strategi för att hantera storleksskillnader i exponering mellan studier (Artikel 4); och en ny enstegsmetod för att estimera komplexa modeller utan att exkludera relevanta studier (Artikel 5).

Artikel 1 beskriver hur huvuddelarna av ovanstående metoder har införts i R-paketet "dosresmeta", tillgängligt via CRAN. De nya funktioner implementerades för att förenkla vissa uppgifter, så som att definiera en designmatris och prediktera sammanslagna resultat. Artikeln illustrerar även beräkningen av linjära och icke-linjära kurvor och utförandet av hypotestester. Utöver detta presenteras resultat från återanalyserade metaanalyser med sammanslagen dos-responsdata.

Artikel 2 utvärderar bedömningen av goodness-of-fit-testet. Den föreslagna metoden består dels av beräkning av deskriptiva mätvärden för att summera likheter och skillnader mellan predikterade och observerade data (avvikelse- och bestämningskoefficient), och dels av grafiska verktyg för att visualisera den predikterade modellen. En åter analys av publicerade metaanalyser exemplifierar hur metoden kan användas för att förbättra kvantitativ syntes av sammanslagen dos-responsdata.

Artikel 3 presenterar ett nytt mått (Rb) för att kvantifiera andelen varians i den sammanslagna skattningen som kan förklaras av heterogeniteten mellan olika studier. I motsats till tidigare mått på heterogenitet kräver Rb varken något antagande om fördelningen av inom-studiefelvariationer eller någon specifikation av deras värden. Resultatet av det föreslagna måttet har utvärderats i en omfattande simuleringsstudie samt genom återanalys av publicerade metaanalyser.

Artikel 4 beskriver hur vi har vidareutvecklat ett punktvist tillvägagångssätt för metaanalys med sammanslagen dos-responsdata. Metoden kombinerar predikterade relativa risker för finfördelade exponeringsvärden baserat på potentiellt olika dos-responsmodeller. Ett punktvist tillvägagångssätt kan förbättra flexibiliteten i modelleringen av studiespecifika kurvor, och minska påverkan från extrapolering, genom att prediktera studiespecifika relativa risker baserat på observerad exponeringsstorlek. 

Artikel 5 presenterar en enstegsmetod för att estimera dos-respons metaanalys för linjära mixade modeller, vilket vanligtvis utförs som en tvåstegsmetod. Fördelarna med en enstegsmetod är många, så som underlättad bedömning av exponeringsheterogenitet och modellskillnader samt förbättrad prediktion av individuella dos-responssamband. 

Sammanfattningsvis utökar och förbättrar metoderna i denna avhandling de tillgängliga verktygen för dos-respons metaanalys. Metoderna har illustrerats med hjälp av befintlig data och är implementerade i det lättillgängliga R-paket.


\normalsize