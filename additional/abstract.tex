% ---------------------------------------------------------
% Project: PhD KAPPA
% File: abstract.tex
% Author: Andrea Discacciati
%
% Purpose: Abstract frontmatter
% ---------------------------------------------------------


\noindent {\Large \textsf{\textbf{{Abstract}}}
\bigskip
%\chapter*{Abstract}
\small
\par Dose-response meta-analysis is a statistical technique increasingly used to combine and contrast the evidence on the association between a continuous exposure and the risk of a health outcome. Several papers refined selected aspects of the methodology, such as implementations of flexible strategies and considering extensions to multivariate meta-analysis. However, there were still several relevant questions that needed to be addressed. This thesis aims to address these issues by developing and implementing (\citetalias{crippadosresmeta2016}) new strategies and ad-hoc measures, including tools for evaluating the goodness-of-fit (\citetalias{discacciati2015goodness}), a new measure for quantifying the impact of heterogeneity (\citetalias{crippa2016new}), a strategy to deal with differences in the exposure range across studies (\citetalias{crippa2018pointwise}), and a one-stage approach to estimate complex models without excluding relevant studies (\citetalias{crippa2018one}).

In \citetalias{crippadosresmeta2016}, we described the implementation of the main aspects of the methodology in the \texttt{dosresmeta} $\R$ package available on CRAN. Dedicated functions was written to facilitate specific tasks such as definition of the design matrix and prediction of the pooled results. We illustrated how to estimate both linear and non-linear curves, conduct test of hypotheses, and present the results in a grapichal way using summarized data on alcohol intake and colorectal cancer risk.
 
In \citetalias{discacciati2015goodness}, we discussed how to evaluate the goodness-of-fit for dose-response meta-analysis. The proposed solutions consist of descriptive measures to summarize the agreement between fitted and observed data (the deviance and the coefficient of determination), and graphical tools to visualize the fit of the model (decorrelated residuals-versus-exposure plot). Data from two published meta-analyses were used to show how these tools can improve the practice of quantitative synthesis of aggregated dose-response data. 

In \citetalias{crippa2016new}, we proposed and discussed a new measure, $R_b$, to quantify the proportion of the variance of the pooled estimate attributable to the between-study heterogeneity. $R_b$ does not make any assumption about the distribution of the within-study error variances, nor does it require specification of a typical value for these quantities. The performance of the proposed measure was evaluated in an extensive simulation study. We demonstrated how to present and interpret the $R_b$ re-analyzing three published meta-analyses.

In \citetalias{crippa2018pointwise}, we extended a point-wise approach to dose--response meta-analysis of aggregated data. The strategy consists of combining predicted relative risks for a fine grid of exposure values based on potentially different dose-response models. Thus, a point-wise approach can improve the flexibility in modelling the study-specific curves and may limit the impact of extrapolation by predicting the study-specific relative risk based on the observe exposure range. We illustrated the methodology using both individual and aggregated participant data.

In \citetalias{crippa2018one}, we formalized a one-stage approach for dose--response meta-analysis in terms of a linear mixed model. We explained the main aspects of the methodology and how to extend the measures typically presented in a two-stage analysis. Using both hypothetical and real data, we showed how the one-stage approach can facilitate investigation the impact of heterogeneity over the exposure range, model comparison, and prediction of individual dose-response associations. The main advantage is that complex curves can be estimated without excluding relevant studies.

In conclusion, the methods presented in this thesis enrich the set of tools available to apply dose--response meta-analyses and to address specific questions including goodness-of-fit evaluation (\citetalias{discacciati2015goodness}) and a new measure of heterogeneity (\citetalias{crippa2016new}). In addition, we presented alternative models for pooling results in case of heterogeneous exposure range (\citetalias{crippa2018pointwise}) and for estimating complex models without excluding relevant studies (\citetalias{crippa2018one}). The proposed methods have been illustrated using real data and implemented in the \texttt{dosresmeta} and \texttt{hetmeta} $\R$ packages available on CRAN (\citetalias{crippadosresmeta2016}). 

\normalsize